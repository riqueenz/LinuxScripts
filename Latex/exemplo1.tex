%% LyX 2.3.6 created this file.  For more info, see http://www.lyx.org/.
%% Do not edit unless you really know what you are doing.
\documentclass[english]{article}
\usepackage[T1]{fontenc}
\usepackage[latin9]{inputenc}
\usepackage{amsbsy}
\usepackage{babel}
\begin{document}
Um tubo de $\boldsymbol{3\frac{1}{2}"}$ \textbf{(88,9 mm)} \'e usado
para fazer a camisa do cilindro. Assim, a press�o necess�ria para
atingir as \textbf{10 toneladas-for\c{c}a }\'e\textbf{ 164,2 bar}.
Veja o c�lculo abaixo:

$A=\frac{\pi.D^{2}}{4}$ | $A=\frac{\pi.(88,9\,mm)^{2}}{4}$ | $\boldsymbol{A=6207,2\,mm^{2}=62,072\,cm^{2}}$

$P=\frac{F}{A}$ | $P=\frac{10000\,kgf}{62,072\,cm^{2}}$ | $\boldsymbol{P=161,1\,kgf/cm^{2}=164,2\,bar}$

Como a press�o nominal (164,4 bar) � menor a press�o admiss�vel (180
bar), ent�o o tubo pode ser esse.

Com 210 bar, a for�a desenvolvida \'e 13,3 toneladas, conforme o
c\'a lculo abaixo:

$P=\frac{F}{A}$ | $F=P.A$ | $F=P.\frac{\pi.D^{2}}{4}$ | $F=210\,bar.\frac{\pi.(88,9\,mm)^{2}}{4}$
| $F=21\,\frac{N}{mm^{2}}.\frac{\pi.(88,9\,mm)^{2}}{4}$ 

\textbf{F = 130350 N = 13,288 toneladas-for\c{c}a}

------------------------------------------------------------------------------------------

Para calcular a vaz\~ao, vamos primeiro calcular o volume do cilindro.

$V=A.S$, sendo S o curso do cilindro. Sabendo que o curso \'e de
$200\,mm$, podemos calcular o volume interno do cilindro:

$V=6207,2\,mm^{2}.200\,mm$ | $\boldsymbol{V=1241433\,mm^{3}=1,24\,litros}$

Desejamos que o cilindro avance em $4\,s.$ Assim, a velocidade necess\'aria
\'e de:

$v=\frac{S}{T}$ | $v=\frac{200\,mm}{4\,s}$| $\boldsymbol{v=50\,mm/s}$

A vaz�o da bomba deve ser de:

$Q=\frac{V}{T}$ | $Q=\frac{1,24\,litros}{4\,s}$ | $Q=0,31\,\frac{litros}{s}.(\frac{60\,s}{1\,min})$
| $\boldsymbol{Q=18,6\,litros/min}$

Podemos ainda usar a \'Area e a velocidade:

$Q=A.v$ | $Q=6207,2\,mm^{2}.50\frac{mm}{s}$ | $Q=310360\frac{mm^{3}}{s}.(\frac{1\,litro}{1000000\,mm^{3}}.\frac{60\,s}{1\,min})$
| $\boldsymbol{Q=18,6\,litros/min}$
\end{document}
